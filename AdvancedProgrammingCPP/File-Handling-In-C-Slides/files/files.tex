\documentclass[17pt]{beamer} 
\usepackage{beamerthemesplit}
\definecolor{White}{RGB}{255,255,255}
\definecolor{Blue}{RGB}{102,51,123}
\setbeamercolor{structure}{fg=Blue}
\setbeamercolor{alerted text}{fg=Blue}
\beamertemplateshadingbackground{Blue!20}{White!20} 
\logo{\includegraphics[height=1cm]{3t-logo.pdf}}
\setbeamersize{text margin left=0cm, text margin right=0cm}
\begin{document}
\sffamily \bfseries
\title
[Files in C\hspace{0.75cm}]
%\insertframenumber/\inserttotalframenumber]
{Files in C}
\author
[Ashwini R Patil]
{\small Talk to a Teacher \\http://spoken-tutorial.org \\ National Mission on Education
  through ICT \\ http://sakshat.ac.in \\ [0.6cm]
   Ashwini R Patil \\IIT Bombay \\[0.6cm]
{\small 20 June 2012}
}
\begin{frame}
   \titlepage
\end{frame}
\begin{frame}[fragile]
  \frametitle{Learning Objectives}\pause
  \begin{itemize}[<+-|alert@+>]
  \item To open a file
  \item To read data from a file
  \item To write data into a file
  \item Examples
  \end{itemize}
\end{frame}
					%%Operating Systems Used
\begin{frame}
  \frametitle {System Requirements} \pause
  \begin{itemize}[<+-|alert@+>]
  \item Ubuntu OS v. 11.10
  \item gcc Compiler v. 4.6.1
  \end{itemize}
\end{frame}


\begin{frame}
  \frametitle{Introduction}\pause
  \begin{itemize}[<+-|alert@+>]
  \item File is a collection of data
  \item It can be a database, a program, a letter or anything
  \item We can create a file and access it using C
 \end{itemize}
\end{frame}

\begin{frame}
  \frametitle{Syntax}\pause
  %\vspace{-0.1in}
  \begin{itemize}[<+-|alert@+>]
  \item {\color{magenta}fp = fopen(``filename", ``mode");}
  \end{itemize}
  \end{frame}
  
  \begin{frame}
  \frametitle{Modes}\pause
  \begin{itemize}[<+-|alert@+>]
 \item {\color{magenta}w} - creates file for read and write
 \item {\color{magenta}r} - opens file for reading
 \item {\color{magenta}a} - writing at the end of file
  \end{itemize}
  \end{frame}


\begin{frame}
\frametitle{Summary}
\begin{itemize}
\item File handling \vspace{0.30cm}
\item To write data into a file \vspace{0.30cm}
\begin{itemize}
\item eg. fp = fopen(``Sample.txt”, ``w"); \vspace{0.30cm}
\end{itemize}
\item To read data from a file \vspace{0.30cm}
\begin{itemize}
\item eg. fp = fopen(``Sample.txt”, ``r"); \vspace{0.30cm}
\end{itemize}
\end{itemize}
\end{frame}

		%Assignment
\begin{frame} 
\frametitle{Assignment}
\begin{itemize}
\item Write a program to create a file TEST
\item Write your name and address in the file TEST
\item Then display it on the console using C program
\end{itemize}
\end{frame}
	% About the Spoken Tutorial project

\begin{frame}
\frametitle{About the Spoken Tutorial Project}
\begin{itemize}
\item Watch the video available at {\color{blue} http://spoken-tutorial.org /What\_is\_a\_Spoken\_Tutorial} 
\item It summarises the Spoken Tutorial project 
\item If you do not have good bandwidth, you can download and watch it
\end{itemize}
\end{frame}

\begin{frame}
\frametitle{Spoken Tutorial Workshops}The Spoken Tutorial Project Team 
\begin{itemize}
\item Conducts workshops using spoken tutorials 
\item Gives certificates to those who pass an online test 
\item For more details, please write to \\ \hspace {0.5cm}{\color{blue}contact@spoken-tutorial.org}
\end{itemize}
\end{frame}

\begin{frame}
\frametitle{Acknowledgements}
\begin{itemize}
\item Spoken Tutorial Project is a part of the Talk to a Teacher  project 
\item It is supported by the National Mission on Education through  ICT, MHRD, Government of India 
\item More information on this Mission is available at: \\ \hspace{0.2cm}{\color{blue} http://spoken-tutorial.org \\ \hspace{0.2cm}/NMEICT-Intro}
\end{itemize}
\end{frame}
\end{document}
